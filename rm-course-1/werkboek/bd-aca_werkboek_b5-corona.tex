% !TEX encoding = UTF-8 Unicode

\documentclass[a4paper,10pt]{report}
\usepackage[natbibapa,nosectionbib,tocbib,numberedbib]{apacite}
\AtBeginDocument{\renewcommand{\bibname}{Literature}}


\usepackage[utf8x]{inputenc}
\usepackage{graphicx}
\usepackage{enumerate}
\usepackage{url}

\usepackage[colorinlistoftodos]{todonotes}

\usepackage{pifont}

\usepackage{lmodern}
\usepackage{listings}
\lstset{
basicstyle=\scriptsize\ttfamily,
columns=flexible,
breaklines=true,
numbers=left,
%stepsize=1,
numberstyle=\tiny,
backgroundcolor=\color[rgb]{0.85,0.90,1}
}


\let\oldquote\quote
\let\endoldquote\endquote
\renewenvironment{quote}{\footnotesize\oldquote}{\endoldquote}



\title{Big Data and Automated Content Analysis\\(6 ECTS) \\~\\Course Manual}
\author{dr. Damian Trilling\\~\\Graduate School of Communication\\University of Amsterdam\\~\\d.c.trilling@uva.nl\\@damian0604}
\date{Academic Year 2019/20\\Semester 2, block 2}


\begin{document}
\maketitle

%\tableofcontents


\chapter{About this course}

This course manual contains general information, guidelines, rules and schedules for the Research Master course Big Data \& Automated Content Analysis. Please make sure you read it carefully, as it  contains information regarding assignments, deadlines and grading.

\section{Course description}
 
``Big data'' is a relatively new phenomenon, and refers to data that are more voluminous, but often also more unstructured and dynamic, than traditionally the case. In Communication Science and the Social Sciences more broadly, this in particular concerns research that draws on Internet-based data sources such as social media, large digital archives, and public comments to news and products This emerging field of studies is also called \emph{Computational Social Science} \citep{Lazer2009} or even \emph{Comutational Communication Science} \citep{Shah2015}.

The course will provide insights in the basic concepts, challenges and opportunities associated with data so large that traditional research methods (like manual coding) cannot be applied any more and traditional inferential statistics start to loose their meaning. Participants are introduced to strategies and techniques for capturing and analyzing digital data in communication contexts, through concrete examples and templates than can be shared and modified for the students’ own research projects. We will focus on (a) data harvesting, storage, and preprocessing and (b) computer-aided content analysis, including natural language processing (NLP) and computational social science approaches.

To participate in this course, students are expected to be interested in learning how to write own programs where off-the-shelf software is not available. Some basic understanding of programming languages is helpful, but not necessary to enter the course. Students without such knowledge are encouraged to follow the (free) online course at \url{https://www.codecademy.com/learn/python} to prepare.

\section{Goals}
Upon completion of this course, the following goals are reached:
\begin{enumerate}[A]
\item Students can explain the research designs and methods employed in existing research articles on Big Data and automated content analysis.
\item Students can on their own and in own words critically discuss the pros and cons of research designs and methods employed in existing research articles on Big Data and automated content analysis; they can, based on this, give a critical evaluation of the methods and, where relevant, give advice to improve the study in question.
\item Students can identify research methods from computer science and computational linguistics which can be used for research in the domain of communication science; they can explain the principles of these methods and describe the value of these methods for communication science research.
\item Students can on their own formulate a research question and hypotheses for own empirical research in the domain of Big Data.
\item Students can on their own chose, execute and report on advanced research methods in the domain of Big Data and automatic content analysis.
\item Students know how to collect data with scrapers, crawlers and APIs; they know how to analyze these data and to this end, they have basic knowledge of the programming language Python and know how to use Python-modules for communication science research.
\item Students can critically discuss  strong and weak points of their own research and suggest improvements.
\item Students participate actively: reading the literature carefully and on time, completing assignments carefully and on time, active participation in discussions, and giving feedback on the work of fellow students give evidence of this.
\end{enumerate}

\section{Help with practical matters}
While making your first steps with programming in Python, you will probably have a lot of questions. During the lab sessions, I will help you with questions. 
Nevertheless, \url{htttp://google.com} and \url{http://stackoverflow.com} should be your first points of contact. After all, that's how we solve our problems as well\ldots


\chapter{Rules, assignments, and grading}
The final grade of this course will be composed of the grade of one mid-term take home exam (30\%) and one individual project (70\%).

\section{Mid-term take-home exam (30\%)}
In a mid-term take-home exam, students will show their understanding of the literature and prove they have gained new insights during the lectures and lab sessions. They will be asked to critically assess various approaches to Big Data analysis and make own suggestions for research.

\section{Final individual project (70\%)}
The final individual project typically consists of the following elements:
\begin{itemize}
\item introduction including references to relevant (course) literature, an overarching research question plus subquestions and/or hypotheses (1–2 pages);
\item an overview of the analytic strategy, referring to relevant methods learned in this course;
\item carefully collected and relevant dataset of non-trivial size;
\item a set of scripts for collecting, preprocessing, and analyzing the data. The scripts should be well-documented and tailored to the specific needs of the own project;
\item output files;
\item a well-substantiated conclusion with an answer to the RQ and directions for future research.
\end{itemize}

\section{Grading and 2\textsuperscript{nd} try}
Students have to get a pass (5.5 or higher) for both the mid-term take-home exam and the individual project. If the grade of one of these is lower, an improved version can be handed in within one week after the grade is communicated to the student. If the improved version still is graded lower than 5.5, the course cannot be completed. Improved versions of the final individual project cannot be graded higher than 6.0. 

\section{Presence and participation}
Attendance is compulsory. Missing more than two of the sixteen meetings – for whatever reason – means the course cannot be completed.

Next to attending the meetings, students are also required to prepare the assigned literature and to continue working on the programming tasks after the lab sessions. To successfully finish the course, attending the lab sessions is not enough, but has to go hand-in-hand with continuos self-study.

\section{Staying informed}
It is your responsibility to check the means of commonications used for this course (i.e., your email account, but – if applicable – also e-learning platforms or any other tool that the lecturer decides to use) on a regular basis, which in most cases means daily.

\section{Plagiarism \& fraud}
Plagiarism is a serious academic violation. Cases in which students use material such as online sources or any other sources in their written work and present this material as their own original work without citation/referencing, and thus conduct plagiarism, will be reported to the Examencommissie of the Department of Communication without any further negotiation. If the committee comes to the conclusion that a student has indeed committed plagiarism the course cannot be completed. 

General UvA regulations about fraud and plagiarism apply.

\section{Deadlines and handing in}
Please send all assignments and papers as a PDF file to ensure that it can be read and is displayed the same way on any device. Hardcopies are not required. Multiple files should be compressed and handed in as one .zip file or .tar.gz file. Anything exceeding a reasonable file size (approximately 5 MB) has to be send via \url{https://filesender.surf.nl/} instead of direct email.

Final papers and take-home exams that are not handed in on time, will be not be graded and receive the grade 1. This rule also applies for any other assignment that might be given. The deadline is only met when the all files are submitted.



 
\chapter{Schedule and Literature}


The following schedule gives an overview of the topics covered each week, the obligatory literature that has to be studied each week, and other tasks the students have to complete in preparation of the class.
In particular, the schedule shows which chapter of \cite{cssbook} will be dealt with. Note that some basic chapters that explain how to install the software we are going to use have to be read before the course starts.

Next to the obligatory literature, the following books provide the interested student with more and deeper information. They are intended for the advanced reader and might be useful for final individual projects, but are by no means required literature. Bear in mind, though, that you may encounter slightly outdated examples (e.g., Python 2, now-defunct APIs etc.).

\begin{itemize}
%\item \citealp{Russel2013}. Gives a lot of examples about how to analyze a variety of online data, including Facebook and Twitter, but going much beyond that.
%\item \citealp{Bird2009}. This is the official documentation of the NLTK package that we are using. A newer version of the book can be read for free at \url{http://nltk.org}
\item \citealp{McKinney2012}: A lot of examples for data analysis in Python. A PDF of the book can be downloaded for free on \url{http://it-ebooks.info/book/1041/}.
\item \citealp{VanderPlas2016}: A more recent book on numpy, pandas, scikit-learn and more. It can also be read online for free on \url{https://jakevdp.github.io/PythonDataScienceHandbook/}, and the contents are avaibale as Jupyter Notebooks as well \url{https://github.com/jakevdp/PythonDataScienceHandbook}.
\item The pandas cookbook by Julia Evans, a collection of notebooks on github: \url{https://github.com/jvns/pandas-cookbook}.
\item \citealp{Hovy2020}: A thin book on bottom-up text analysis in Python with both a bit more math background and ready-to-use Python code implementations.
\item \citealp{Salganik2017}: Not a book on Python, but on research methods in the digital age. Very readable, and a lots of inspiration and background about techniques covered in our course.
\end{itemize}


\section*{Course setup in times of Corona}
Usually, each week of this follow the following structure: The first meeting is a lecture, the second meeting is a lab session during which students work through the material, do exercises, and a teacher is around for answering questions, helping, and explaining again what has been difficult. The latter is very hard, if not impossible, to do in a non-face-to-face setting, as it centers around walking around, watching students code, etc.

To make it an as useful as possible learning experience for everyone, we will use the following structure this time:

\begin{itemize}
\item The lectures (usually: Mondays) stay the same and will be delivered through Big Blue Button on Canvas.
\item The students are expected to work through the chapters and do all associated exercises not during the lab sessions, but before (i.e., on Monday, Tuesday, or Wednesday).
\item Until the day before the lab session, 15.00 (i.e., usually, Wednesday, 15.00), students can post questions and problems they encounter on a forum on Canvas.
\item The lab session itself will then be replaced by a lecture in which the teacher will answer the questions that have been posted, prioritzing those questions that cannot easily be found in the study material or similar sources.
\end{itemize}


\section*{Before the course starts: Prepare your computer.}
\textsc{\ding{52} Chapter 1: Preparing your computer}\\
Follow all steps as outlined in Chapter 1 \emph{or} install Anaconda as outlined in the Appendix.


\section*{Week 1: Introduction}
\subsection*{Monday, 30--3. Lecture.}
We discuss what Big Data means, how the concept can be understood, what challenges and opportunities arise, and what the implications are for communication science. 

Mandatory readings (in advance): \cite{boyd2012} and \cite{Kitchin2014}. 

Additional literature, not obligatory to read in advance, but very informative: \cite{Mahrt2013}, \cite{Vis2013}, \cite{Trilling2017a}.



\subsection*{Thursday, 2--4. Online lab session.}

Prepare and ask questions in advance about:\\
\textsc{\ding{52} Chapter 2: The Linux command line}\\
\textsc{\ding{52} Chapter 3: A language, not a program}\\

Also, make sure that you can a basic program in Python, such as \texttt{print('Hello World')} in multiple environments, such as Jupyer Notebook or Spyder.


\section*{Week 2: Getting started with Python}

\subsection*{Monday, 6--4. Lecture.}
\textsc{\ding{52} Chapter 4: The very, very basics of programming in Python}\\
You will get a very gentle introduction to computer programming. During the lecture, you are encouraged to follow the examples on your own laptop.


\subsection*{Thursday, 9--4. Lab session.}
Prepare and ask questions in advance about:\\
\textsc{\ding{52} Appendix A: Exercise 1}\\



\section*{Week 3: Data harvesting and storage}
This week is about data sources and their (dis)advantages. 

\subsection*{Monday, 13--4. No meeting (Easter)}
\subsection*{Tuesday (!!!), 14--4. Lecture}
A conceptual overview of APIs, scrapers, crawlers, RSS-feeds, databases, and different file formats.

Read the article by \cite{Morstatter2013} in advance. It discusses the quality of data provided by the Twitter API. As a practical example for how ``dirty'' input data (i.e., data that for whatever reason does not come in form of a clean, structured data set like a table) can be parsed and preprocessed, have a look at the method section of the article by \cite{Lewis2013}. 


\subsection*{Thursday, 16 April. Lab session}
Prepare and ask questions in advance about:
\textsc{\ding{52} Chapter 5.1--5.4: Retrieving and storing data}\\




\section*{Week 4: Sentiment analysis.}
Up till now, we have mainly talked about available data and how to acquire them. From now on, we will focus on analyzing them and cover one technique per week. By now, you should also have gotten some idea about your final project.


\subsection*{Monday, 20---4. Lecture.}
We start with an overview of different analytical approaches which we will cover in the next weeks, After that, we will focus on the first of these techniques, sentiment analysis.

Mandatory readings (in advance): \cite{GonzalezBailon2015},  \cite{Hutto2014}, and \cite{Vermeer2019}.

Suggestions for additional readings:
\begin{itemize}
	\item Examples of (simple) sentiment analyses: \cite{Huang2007,Pestian2012, Mostafa2013}. 
	\item If you want to have a look under the hood of another popular sentiment analysis algorithm, you can read \cite{Thelwall2012}.
\end{itemize}



\subsection*{Take-home exam}
In week 4, the first midterm take-home exam is distributed after the Monday meeting. The answer sheets and all files have to be handed in no later than the day before the next meeting, i.e. Wednesday evening (22--4, 23.59).



\subsection*{Thursday, 23--4. Lab session.}
As you are working on your take-home exam, I do not expect you to prepare questions in this week (even though you \emph{can} ask them if you want). Instead, I will prepare answers to frequently asked questions.\\
\textsc{\ding{52} Chapter 6: Sentiment analysis}\\




\section*{Week 5: Automated content analysis with NLP and regular expressions.}
Text as written by humans usually is pretty messy. You will learn how to process text to make it suitable for further analysis by using techniques of Natural Language Processing (NLP), and how to extract meaningful information (discarding the rest) using regular expressions. You will also make a first aquintance with the packages NLTK and spacy.


\subsection*{Monday, 27--4. No meeting (Koningsdag)}
\subsection*{Tuesday (!!!), 28--4. Lecture}
This lecture will introduce you to techniques and concepts like stemming, stopword removal, n-grams, word counts and word co-occurrances, and regular expressions. We will do some exercises during the lecture.

Preparation: Mandatory reading: \cite{Boumans2016}. 


\subsection*{Thursday, 30--4. Lab session.}
Prepare and ask questions in advance about:
\textsc{\ding{52} Chapter 7: Automated content analysis}\\




\section*{Week 6: Web scraping and parsing}

\subsection*{Monday, 4--5. No meeting (Dodenherdenking)}


\subsection*{Thursday, 7--5. Lecture.}
\textsc{\ding{52} Chapter 8: Web scraping}\\
We will explore techniques to download data from web pages and to extract meaningful information like the text (or a photo, or a headline, or the author) from an article on \url{http://nu.nl}, a review (or a price, or a link) from \url{http://kieskeurig.nl}, or similar. 

Try to write a web scraper at home and post questions before the next lecture.




\section*{Week 7: Statistics with Python}

\subsection*{Monday, 11--5. Short lecture plus lab session.}
\textsc{\ding{52} Section 3.5: Jupyter Notebook}\\
\textsc{\ding{52} Chapter 12: Statistics with Python}\\
You have worked hard so far, so we'll do something fun and relaxing (of course, fun might be a relative concept in this course\ldots). You are going to learn how to create visualizations, do conventional statistical tests, manage datasets with Python, save the results together with your code and your own explanations -- and all of this within your browser.

We will learn how to do data wrangling with pandas: converting between wide and long formats (melting and pivoting), aggregating data, joining datasets, and so on.

We will also reserve some time for questions regarding last week.


\subsection*{Thursday, 14--5. Lab session}
Prepare any questions that you may have about any topic that you may have about any topic in this course, especially regarding techniques you need for your final project.



\subsection*{Final project}
Deadline for handing in: Wednesday, 29--5, 23.59.




\bibliographystyle{apacite}
\bibliography{../../bdaca}

 
 
 
\end{document}
