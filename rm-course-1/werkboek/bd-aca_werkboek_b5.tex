% !TEX encoding = UTF-8 Unicode

\documentclass[a4paper,10pt]{report}
\usepackage[natbibapa,nosectionbib,tocbib,numberedbib]{apacite}
\AtBeginDocument{\renewcommand{\bibname}{Literature}}


\usepackage[utf8x]{inputenc}
\usepackage{graphicx}
\usepackage{enumerate}
\usepackage{url}

\usepackage[colorinlistoftodos]{todonotes}

\usepackage{pifont}

\usepackage{lmodern}
\usepackage{listings}
\lstset{
basicstyle=\scriptsize\ttfamily,
columns=flexible,
breaklines=true,
numbers=left,
%stepsize=1,
numberstyle=\tiny,
backgroundcolor=\color[rgb]{0.85,0.90,1}
}


\let\oldquote\quote
\let\endoldquote\endquote
\renewenvironment{quote}{\footnotesize\oldquote}{\endoldquote}



\title{Big Data and Automated Content Analysis\\~\\Course Manual}
\author{dr. Damian Trilling\\~\\Graduate School of Communication\\University of Amsterdam\\~\\d.c.trilling@uva.nl\\www.damiantrilling.net\\@damian0604\\~\\Office: REC-C, 8\textsuperscript{th} floor}
\date{Academic Year 2017/18\\Semester 2, block 2}


\begin{document}
\maketitle

%\tableofcontents


\chapter{About this course}

This course manual contains general information, guidelines, rules and schedules for the Research Master course Big Data \& Automated Content Analysis. Please make sure you read it carefully, as it  contains information regarding assignments, deadlines and grading.

\section{Course description}
 
``Big data'' is a relatively new phenomenon, and refers to data that are more voluminous, but often also more unstructured and dynamic, than traditionally the case. In Communication Science and the Social Sciences more broadly, this in particular concerns research that draws on Internet-based data sources such as social media, large digital archives, and public comments to news and products This emerging field of studies is also called \emph{Computational Social Science} \citep{Lazer2009} or even \emph{Comutational Communication Science} \citep{Shah2015}.

%One of the big challenges is being able to derive information from these data that can be handled meaningfully and economically at the same time.

The course will provide insights in the basic concepts, challenges and opportunities associated with data so large that traditional research methods (like manual coding) cannot be applied any more and traditional inferential statistics start to loose their meaning. Participants are introduced to strategies and techniques for capturing and analyzing digital data in communication contexts, through concrete examples and templates than can be shared and modified for the students’ own research projects. We will focus on (a) data harvesting, storage, and preprocessing and (b) computer-aided content analysis, including natural language processing (NLP) and computational social science approaches.

To participate in this course, students are expected to be interested in learning how to write own programs where off-the-shelf software is not available. Some basic understanding of programming languages is helpful, but not necessary to enter the course. Students without such knowledge are encouraged to follow the (free) online course at \url{https://www.codecademy.com/learn/python} to prepare.

\section{Goals}
Upon completion of this course, the following goals are reached:
\begin{enumerate}[A]
\item Students can explain the research designs and methods employed in existing research articles on Big Data and automated content analysis.
\item Students can on their own and in own words critically discuss the pros and cons of research designs and methods employed in existing research articles on Big Data and automated content analysis; they can, based on this, give a critical evaluation of the methods and, where relevant, give advice to improve the study in question.
\item Students can identify research methods from computer science and computational linguistics which can be used for research in the domain of communication science; they can explain the principles of these methods and describe the value of these methods for communication science research.
\item Students can on their own formulate a research question and hypotheses for own empirical research in the domain of Big Data.
\item Students can on their own chose, execute and report on advanced research methods in the domain of Big Data and automatic content analysis.
\item Students know how to collect data with scrapers, crawlers and APIs; they know how to analyze these data and to this end, they have basic knowledge of the programming language Python and know how to use Python-modules for communication science research.
\item Students can critically discuss  strong and weak points of their own research and suggest improvements.
\item Students participate actively: reading the literature carefully and on time, completing assignments carefully and on time, active participation in discussions, and giving feedback on the work of fellow students give evidence of this.
\end{enumerate}

\section{Help with practical matters}
While making your first steps with programming in Python, you will probably have a lot of questions. During the lab sessions, next to me (Damian), Joanna Strycharz (j.strycharz@uva.nl) will be present to help with questions. 
Nevertheless, \url{htttp://google.com} and \url{http://stackoverflow.com} should be your first points of contact. After all, that's how we solve our problems as well\ldots


\chapter{Rules, assignments, and grading}
The final grade of this course will be composed of the grade of one mid-term take home exam (30\%) and one individual project (70\%).

\section{Mid-term take-home exam (30\%)}
In a mid-term take-home exam, students will show their understanding of the literature and prove they have gained new insights during the lectures and lab sessions. They will be asked to critically assess various approaches to Big Data analysis and make own suggestions for research.

\section{Final individual project (70\%)}
The final individual project typically consists of the following elements:
\begin{itemize}
\item introduction including references to relevant (course) literature, an overarching research question plus subquestions and/or hypotheses (1–2 pages);
\item an overview of the analytic strategy, referring to relevant methods learned in this course;
\item carefully collected and relevant dataset of non-trivial size;
\item a set of scripts for collecting, preprocessing, and analyzing the data. The scripts should be well-documented and tailored to the specific needs of the own project;
\item output files;
\item a well-substantiated conclusion with an answer to the RQ and directions for future research.
\end{itemize}

\section{Grading and 2\textsuperscript{nd} try}
Students have to get a pass (5.5 or higher) for both the mid-term take-home exam and the individual project. If the grade of one of these is lower, an improved version can be handed in within one week after the grade is communicated to the student. If the improved version still is graded lower than 5.5, the course cannot be completed. Improved versions of the final individual project cannot be graded higher than 6.0. 

\section{Presence and participation}
Attendance is compulsory. Missing more than two of the sixteen meetings – for whatever reason – means the course cannot be completed.

Next to attending the meetings, students are also required to prepare the assigned literature and to continue working on the programming tasks after the lab sessions. To successfully finish the course, attending the lab sessions is not enough, but has to go hand-in-hand with continuos self-study.

\section{Staying informed}
It is your responsibility to check the means of commonications used for this course (i.e., your email account, but – if applicable – also e-learning platforms or any other tool that the lecturer decides to use) on a regular basis, which in most cases means daily.

\section{Plagiarism \& fraud}
Plagiarism is a serious academic violation. Cases in which students use material such as online sources or any other sources in their written work and present this material as their own original work without citation/referencing, and thus conduct plagiarism, will be reported to the Examencommissie of the Department of Communication without any further negotiation. If the committee comes to the conclusion that a student has indeed committed plagiarism the course cannot be completed. 

General UvA regulations about fraud and plagiarism apply.

\section{Deadlines and handing in}
Please send all assignments and papers as a PDF file to ensure that it can be read and is displayed the same way on any device. Hardcopies are not required. Multiple files should be compressed and handed in as one .zip file or .tar.gz file. Anything exceeding a reasonable file size (approximately 5 MB) has to be send via \url{https://filesender.surf.nl/} instead of direct email.

Final papers and take-home exams that are not handed in on time, will be not be graded and receive the grade 1. This rule also applies for any other assignment that might be given. The deadline is only met when the all files are submitted.


 
 
 
\chapter{Schedule and Literature}

The following schedule gives an overview of the topics covered each week, the obligatory literature that has to be studied each week, and other tasks the students have to complete in preparation of the class.
In particular, the schedule shows which chapter of \cite{Trilling2016} will be dealt with. Note that some basic chapters, which provide the students with the computer skills necessary to use our tools and explain which software to install, have to be read before the course starts.

Next to the obligatory literature, the following books provide the interested student with more and deeper information. They are intended for the advanced reader and might be useful for final individual projects, but are by no means required literature:

\begin{itemize}
\item \citealp{Russel2013}. Gives a lot of examples about how to analyze a variety of online data, including Facebook and Twitter, but going much beyond that.
\item \citealp{Bird2009}. This is the official documentation of the NLTK package that we are using. A newer version of the book can be read for free at \url{http://nltk.org}
\item \citealp{McKinney2012}: Another book with a lot of examples. A PDF of the book can be downloaded for free on \url{http://it-ebooks.info/book/1041/}.
\end{itemize}



\section*{Before the course starts: Prepare your computer.}
\textsc{\ding{52} Chapter 1: Preparing your computer}\\
Follow all steps as outlined in Chapter 1.

\section*{Week 1: Introduction}
\subsection*{Wednesday, 4-4. Lecture.}
{\footnotesize{(Damian)}\\}

We discuss what Big Data means, how the concept can be understood, what challenges and opportunities arise, and what the implications are for communication science. 

Mandatory readings (in advance): \citealp{boyd2012}, \citealp{Kitchin2014}. 

Additional literature, not obligatory to read in advance, but very informative: \citealp{Mahrt2013}, \citealp{Vis2013}, \citealp{Trilling2017a}.

%The articles mentioned above discuss the implications of Big Data methods in a very broad way. You should also have a look at some applied articles in the field to get an idea of the type of research that is currently conducted in the field. Good readings are \citealp{Castillo2014,Ellison2013,Conover2012}. You do not have to read all of them in detail, but should get a general understanding of the types of methods that are used in these studies.


\subsection*{Friday, 6--4. Lab session.}
\textsc{\ding{52} Chapter 2: The Linux command line}\\
\textsc{\ding{52} Chapter 3: A language, not a program}\\
{\footnotesize{(Damian and Joanna)}\\}
We will get familiar with the Virtual Machine and the software we will work with. Make sure you installed everything in advance and that you can start up your machine. 





\section*{Week 2: Getting started with Python}

\subsection*{Wednesday, 11--4. Lecture.}
\textsc{\ding{52} Chapter 4: The very, very basics of programming in Python}\\
{\footnotesize{(Damian)}\\}
You will get a very gentle introduction to computer programming. During the lecture, you are encouraged to follow the examples on your own laptop.


\subsection*{Friday, 13--4. Lab session.}
\textsc{\ding{52} Appendix A: Exercise 1}\\
{\footnotesize{(Damian and Joanna)}\\}
We will do our first real steps in Python and do some exercises to get the feeling. 


\section*{Week 3: Data harvesting and storage}
This week is about data sources and their (dis)advantages. 

\subsection*{Wednesday, 18--4. Lecture.}
{\footnotesize{(Damian)}\\}
A conceptual overview of APIs, scrapers, crawlers, RSS-feeds, databases, and different file formats.

Read the article by \cite{Morstatter2013} in advance. It discusses the quality of data provided by the Twitter API. As a practical example for how ``dirty'' input data (i.e., data that for whatever reason does not come in form of a clean, structured data set like a table) can be parsed and preprocessed, have a look at the method section of the article by \cite{Lewis2013}. 


\subsection*{Friday, 20--4. Lab session.}
\textsc{\ding{52} Chapter 5: Retrieving and storing data}\\
{\footnotesize{(Damian and Joanna)}\\}
We will write a script to collect some data. 




\section*{Week 4: Sentiment analysis}
Up till now, we have mainly talked about available data and how to acquire them. From now on, we will focus on analyzing them and cover one technique per week. By now, you should also have gotten some idea about your final project.


\subsection*{Wednesday, 25--4. Lecture.}
{\footnotesize{(Damian)}\\}
We start with an overview of different analytical approaches which we will cover in the next weeks, After that, we will focus on the first of these techniques, sentiment analysis.

Read the following two articles in advance. The first one gives an overview of how to analyze social media data, in this case, Twitter \citep{Bruns2013}. The other one is an example of a sentiment analysis \citep{Mostafa2013}.

Some additional examples of sentiment analysis (not obligatory): \cite{Huang2007,Pestian2012}. If you want to have a look under the hood of a popular sentiment analysis algorithm, you can read \cite{Thelwall2012} and \cite{Hutto2014}.


\subsection*{Friday, 27--4. Holiday (Koningsdag)}




\section*{Week 5: Automated content analysis with NLP and regular expressions}
Text as written by humans usually is pretty messy. You will learn how to process text to make it suitable for further analysis by using techniques of Natural Language Processing (NLP), and how to extract meaningful information (discarding the rest) using regular expressions.



\subsection*{Wednesday, 2--5. Lab session (accompanying Lecture week 4).}
\textsc{\ding{52} Chapter 6: Sentiment analysis}\\
{\footnotesize{(Damian and Joanna)}\\}
You will write a tool to read data and conduct a sentiment analysis.



\subsection*{Friday, 4--5. Lecture with exercises}
\textsc{\ding{52} Chapter 7: Automated content analysis}\\
{\footnotesize{(Damian)}\\}
This lecture will introduce you to techniques and concepts like stemming, stopword removal, n-grams, word counts and word co-occurrances, and regular expressions. We will do some exercises during the lecture.

Preparation: Mandatory reading: \citealp{Boumans2016}. Also read the paper by \cite{Madnani}. It uses the same package (NLTK) which we use in class, If you don’t get all practical details yet, that’s OK. Pay special attention to the (linguistic) concepts applied.  

\subsection*{Take-home exam}
In week 5, the midterm take-home exam is distributed after the Friday meeting. The answer sheets and all files have to be handed in no later than the day before the next meeting, i.e. Tuesday evening (Tuesday, 8--5, 23.59).




\section*{Week 6: Statistics with Python}

\subsection*{Wednesday, 9--5. Short lecture plus lab session.}
\textsc{\ding{52} Section 3.5: Jupyter Notebook}\\
\textsc{\ding{52} Chapter 12: Statistics with Python}\\
{\footnotesize{(Damian and Joanna)}\\}
You have worked hard so far, so we'll do something fun and relaxing (of course, fun might be a relative concept in this course\ldots). You are going to learn how to create visualizations, do conventional statistical tests, manage datasets with Python, save the results together with your code and your own explanations -- and all of this within your browser.



\subsection*{Friday, 11--5. Holiday (Hemelvaart)}




\section*{Week 7: Web scraping and parsing}

\subsection*{Wednesday, 16--5. Lecture.}
{\footnotesize{(Damian)}\\}
We will explore techniques to download data from web pages and to extract meaningful information like the text (or a photo, or a headline, or the author) from an article on \url{http://nu.nl}, a review (or a price, or a link) from \url{http://kieskeurig.nl}, or similar. 

\subsection*{Friday, 18--5. Lab session}
\textsc{\ding{52} Chapter 8: Web scraping}\\
{\footnotesize{(Damian and Joanna)}\\}
We will exercise with web scraping and parsing.



\section*{Week 8: Machine learning}

\subsection*{Wedneday, 23--5. Lecture.}
{\footnotesize{(Damian)}\\}
This lecture will introduce you to one of the most fascinating topics in automated content analysis: machine learning. I will walk you trough the ideas behind unsupervised and supervised machine learning. The nice thing is that you actually have already done it during your studies: Principal component analysis is a form of unsupervised ML and regression analysis a form of supervised ML -- you just never called it like this. And you probably never thought about using these techniques to analyze texts (or images). And that's what we are going to do.

\subsection*{Friday, 25--5. Lab session.}
\textsc{\ding{52} Chapter 10: Supervised machine learning}\\
\textsc{\ding{52} Chapter 11: Unsupervised machine learning}\\
{\footnotesize{(Joanna)}\\}
We will exercise with different forms of machine learning. 


\section*{Week 9: Finish!}
This week is designated to working on your final projects. To assist you, we still have two meetings.

\subsection*{Monday, 30--5. Lecture.}
{\footnotesize{(Damian)}\\}
We will look back and systematize what we have learned. Also, this lecture will leave room for possible additional topics that you became interested in during this course and that haven't been covered extensively enough. %For example, we might dig a bit deeper into some specific form of ML; or we might look more in detail into the possibilities of pandas. Or we might have a look at Chapter~9, which we will otherwise skip.


\subsection*{Friday, 1--6. Open Lab.}
{\footnotesize{(Damian and Joanna)}\\}
Possibility to ask last questions regarding the final project.

\subsection*{Final project}
Deadline for handing in: Sunday, 3--6, 23.59.

\bibliographystyle{apacite}
\bibliography{../../bdaca}

 
 
 
\end{document}